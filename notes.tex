\documentclass{article}
\usepackage[utf8]{inputenc}
\usepackage{amsmath}
\usepackage{amsfonts}
\usepackage{amssymb}
\usepackage{amsthm}
\usepackage{epsfig}
\usepackage{epstopdf}
\usepackage{titling}
\usepackage{url}
\usepackage{array}
\usepackage{enumerate}
\usepackage{ physics }
\usepackage{graphicx}
\usepackage{marginnote}
\usepackage{getfiledate}
\usepackage{wrapfig}
\usepackage{gensymb}
\usepackage{hyperref}
\usepackage[font=small,labelfont=bf]{caption}
\graphicspath{ {/} }
\usepackage[hmargin=3.5cm,vmargin=2.5cm]{geometry}
\title{WQ App Model}
\author{Shehtab Zaman }
\date{May 2018}

\begin{document}
\maketitle
\section{Introduction}

\subsection{Outline of Procedure}
The problem of using variable smartphone cameras for the
spectrophotometry of lake water consists of a set of challenges.
The general outline of the procedure is as follows:
\begin{enumerate}
\item Obtain the raw image data for submerged and unsubmerged secchi disk
\item Obtain the intensities of the captured image using QE Curves
\item Normalize intensity to baseline factoring in shutter speed and aperture
\item Obtain intensity of a particular wavelenth provided RGB value
\item From Beer-Lambert law find the concentration of materials in the water
\end{enumerate}

% \section{Background}
\section{Model}
Given our inputs of submerged and unsubmerged disk images, a model for the concentration
can be created using Fresnels Laws of reflections, Beer Lambert Law and the Rendering Equation.
Due to the non-analytic nature of the inverted rendering equation, we must use an approximation technique
to obtain the concentration. Here the taylor expansion to the first and  second order are obtained,
but numerical approximation must prove to be more stable/accurate.

\subsection{Fresnels Reflection and Transmittance}

If we consider the intensity of the light detected by the sensor, $ L_{sensor}$, we can
effectively decompose the sources into two components: the reflection off of the air-water
interface (air-to-air) $L_{reflected}$, and a proportion of the reflectance off of
the disk underwater, $K \cdot L_{transimitted}$. Where
$ K$ is determined by Rendering equation as the concentration of particulates as a parameter.

Given the ambient atmospheric light intenstiy, $L_{atm}$, by energy considerations we can say
$$L_{atm} = L_{reflected}+L_{transmitted}$$

\subsubsection{Reflectance}
Since the incident light is unpolarized, the reflectance is given
by the average of S-polarized and P-polarized light. As an approximation
for a given incident angle $\theta$, we can use Schlick's approximation to obtain
reflectance R.
$$ R(\theta) = \left(\frac{n_1-n_2}{n_1+n_2}\right)^2 +
 \left(1-\left(\frac{n_1-n_2}{n_1+n_2}\right)^2\right)\left(1-cos(\theta)\right)^5 $$
where, $ n_1$ is the index of refraction of the incident media (air in our case)
and $ n_2$ is the index of refracton of the reflected media (water = 1.33).

Since the reflectance is wavelength independent, we have,

$$ L_{reflected} = R \cdot L_{atm}$$

\subsubsection{Transmittance}
\marginnote{In practice considering the reflection may be trivial as at $ \theta = 0$
the maximum reflectance is $\approx 3\%$ and has a significant effect at higher incident
angles only $(>60 \degree)$.}

We can relate the transmittance $ T$ of the unpolarized light with
the reflectance $ R$ as
$$ T = 1 - R$$

We have the transmitted light into the water as,

$$ L_{transmitted} = T \cdot L_{atm} $$

The attenuation $ L_{transmitted}$ due to chlorophyll absoprtion and reflection
on the disk is discussed in the following sections.



\subsection{Rendering Equation}

The interaction of incident, reflected, and emmited light at any given point on a
given surface is
described by the general Rendering Equation,

\begin{equation}
  L_o = L_e + L_r
\end{equation}
Where, $L_o$ is the light leaving the particular point, $ L_e$ is the light emitted
from the point (if it is a source), and $L_r $ is the light reflected from that point.

Since we will be working with non-light generating disks, we set,
$ L_e = 0$.
The light reflected off of the surface of a disk, can be calculated using the
Bidirectional Reflectance Distribution Function (BRDF)[3].
For a given point on a surface, the general BRDF is,

\begin{equation}
\rho_d(\phi_0, \theta_0,\phi_f, \theta_f)=\frac{L_r{\phi_f, \theta_f}}{L_I{\phi_0, \theta_0}}
\end{equation}
Where $\phi_0, \theta_0$ are the incident azimuthal and polar angle of the light source,
and $\phi_f, \theta_0$ are the azimuthal and polar angle with respect to the viewer and
and relfected point.

From the general rendering equation, the reflected radiance is given by

\begin{equation}
  L_r = \int^{2\pi}_{\phi_i = 0}\int^{\frac{\pi}{2}}_{\theta_i = 0}
  f_r L_s T(d)cos(\theta_i)sin(\theta_i)d\theta_i d\phi_i
\end{equation}

Where, the $ f_r$ is Lambertian BRDF.

The transmission function $T(d)$ can be related to the depth of the reflecting
surface.

\subsection{Beer-Lambert Law}

The Beer-Lambert provides us with a linear relationship betweem incident light, absorbed light,
and concentration of absorbed light.

The wavelength dependent relation is given as,

\begin{equation}
  \epsilon d c = Log_{10}{\frac{I_0}{I_f}}
\end{equation}
where, $\epsilon$ is the molar absorptivity in $ L \cdot cm^{-1} \cdot mol^{-1}$,
$ d$ is the depth in $cm$, $ C$ is the concentration in $mol \cdot L^{-1} $. $ I_0$
is the initial light intensity, and $ I_f$ is the final intensity after
absorption.

Given some incident light of radiance, $L_0$, on the water surface,

we have the decomposed radiance as
$$ L_0 = L_{sa} + L_{s}$$
where, $ L_{sa}$ is the absorbed light radiation. $ L_{s}$ is the light
incident on the point under water and being reflected by the disk.

According to the Beer-Lambert Law we have the relation,

\begin{equation}
  Log_{10}\left(\frac{L_0}{L_{s}}\right) = \frac{\epsilon d C}{cos(\theta)}
\end{equation}
and,
\begin{equation}
  T(d) = 10^\frac{-\epsilon d C}{cos(\theta)}
\end{equation}

\subsection{Reflectance}

Since we want to specifically take into account the
absorbance due to chlorophyll a, we limit the consideration
to considering the radiance of light at $\lambda = 662nm$

Even though the incident light is poloychromatic, but since we are considering the specific wavelength
and the incident light is so spread out, we can reasonably treat it as monochromatic. (?)

The reflectance in the particular frequency then becomes,

\begin{equation}
  L_{r \lambda} = \int^{2\pi}_{\phi_i = 0}\int^{\theta_c}_{\theta_i = 0}
  f_r L_{s\lambda} 10^\frac{-\epsilon d C}{cos(\theta)}cos(\theta_i)sin(\theta_i)d\theta_i d\phi_i
\end{equation}
\marginnote{The cutoff angle should ideally vary with the azimuthal position of the sun, time of day and the lattitude of position.}
Here we sum the incident angle upto some cutoff angle $\theta_c$, as the higher incident lights
have negligible effect on the reflected light.


We can further make an assumption that the secchi disk is perfectly matte with rotationally
invariant reflectance. The Lambertian BRDF is thus,

\begin{equation}
  f_r = \frac{\rho_D}{\pi}
\end{equation}
Where, $ \rho_D$ is the albedo of the secchi disk with $ 0 \leq \rho_D \leq 1$
Combining (4) and (5) we get the reflectance off of the surface of secchi disk as,

\begin{equation}
  L_{r \lambda} = \int^{2\pi}_{\phi_i = 0}\int^{\theta_c}_{\theta_i = 0}
  \frac{\rho_D}{\pi} L_{s\lambda} 10^\frac{-\epsilon d C}{cos(\theta)}cos(\theta_i)sin(\theta_i)d\theta_i d\phi_i
\end{equation}

Chlorophyll a has an extinction coefficient of $ \epsilon_{663} = 0.088 \frac{cm^{-1}}{g}$ at 663nm.
Since the light collected by the camera will have had to travel up the water column we must add
another absorbance factor determined by the equation 2.

The light incident on the camera is therefore,
\begin{align}
  L_{i 663} = \int^{2\pi}_{\phi_i = 0}\int^{\theta_c}_{\theta_i = 0}
  \frac{\rho_D}{\pi}  L_{s 663} 10^\frac{-2\epsilon d C}{cos(\theta)}cos(\theta_i)sin(\theta_i)d\theta_i d\phi_i
\end{align}
\subsection{Analytical Solution}
\subsubsection{At High Cutoff Angles}
At high cutoff angles $ \theta_c > 60 \degree$, we can reasonably account for reflectance off
of the water surface to be sufficiently low $ (\leq 3\%)$. The correction to the incident is
sufficiently low and can be discounted.

Equation (10) can be solved analytically to the first order approximation of the taylor expansion of
$10^{\frac{-\epsilon dC}{cos(\theta_i)}}$, for a light source which is at angle $\theta_0$ to $ \theta_{c}$.
The concentration of the chlorophyll, c, can be retrieved by,
\begin{align}
  C &= \frac{ln \left (\frac{L_{i663}}{2\rho_dL_{s663} \left(\left. \frac{\cos^2(\theta_i)}{2} \right \rvert_{\theta_0}^{\theta_c}\right) } \right)}{\epsilon d ln(10^{-2})}
\end{align}

\begin{center}
  \includegraphics[scale=.60]{FirstOrderConcentration.jpg}
\end{center}
Notice that as $d \rightarrow 0 $ the equation reduces to a function of incident angle and properties
of the material and will correspond to the concentration $ C \rightarrow 0 $. The coupling
of $ \epsilon$ and $d$ also shows that in a particulate free medium $(\epsilon = 0)$, the
reflectance is affected by the depth $ d$ and is analogous to the case where $ d = 0$ for
any depth.

In this case, $ \theta_c$ is less than the $ \frac{\pi}{3}$ so we can assume the
difference transmitted intensity is the same as the incident atmospheric intensity. The addition
of higher incident angles would make equation unsolvable analytically
(But I believe we can approximate using perturbation theory).
\subsection{Second Order approximation}
(I'm still grappling with the math on this one)
The Appendix has the math carried out but the last approximation is troubling me.

The orange line is the correction to the first order approximation in the concentraiton
function. The end points behave as expected.
\begin{center}
\includegraphics[scale=.60]{SecondOrderConcentration.jpg}
\end{center}

\subsection{Possible Concerns}

\begin{itemize}
  \item The issues with non-constant air-water interface (waves) is that both the incident
  light intenstiy and reflected light intensity can very dramatically over the the
  surface of the secchi disk and result in patches of high reflectance and distorted disk
  image. Imposing a periodic boundary condition, by defining the water surface as a plane
  wave could be a better approximation. There would be a significant increase in calculation
  complexity in that case.
  \item Another siginificant ommitance is back scattering from the reflected light, being reflected
  back onto the disk. (For an approximation, I believe the correction due to this would be
  marginal, but I have to carry out a sample calculation to have a concrete value).
\end{itemize}
\section{Aperture}

To gather consistent data from different types of smartphone cameras, we must establish
a baseline for the image intensity to be able to normalize pixel data. The varying focal lengths and
f-number provide a challenge to make the data normalized.

Consider a measurement done using a particular smartphone. The irradiance
or incoming light would be calculated using the aperture area of the sensor.
The irradiance is inversely proportional to the aperture area and thus inverse
squarely proportional to the aperture diameter.

The amount of light is also dependent on the light cone which is also
proportional to the distance to the object. Assuming all the measurements are
done at a constant distance, we can solely focus on the variance in the
sensor aperture.

The paraxial irradiance, $ I_C$ of an image is proportional to $I_c \propto \frac{D^2_A}{F^2} $,
where $ D_A$ is the aperture diameter and $ F$ is the focal length of the sensor.

The ratio of $ F$ and $ D_A$ is defined as the relative aperture and we can use it
to relate different sensors with differing sensor focal lenghts.

\subsection{Relative Aperture}

The finite size of a lens limits the amount of light it can collect at ny given time.
Higher integration times (discussed later) offer linear responce up to a certain
time (Schwarzchild Limit).

The exposure depends on thing time t and the irradiance $ I_C$
of the light falling onto a sensor location at the image plane C.
The amount of fluc (photon lines) admitted into the sensor zone is propotional
to the area of the aperture A and thefore the square of the diameter $ D_A$.

A point O of an image is one apex
of a solid angle cone whose base is the system aperture area. Therefore a larger
diameter increases the base of the flux cone and increases the amount of flux wihin the cone
collected by the lens.

Since a higher focal length also increases the length of
the cone and therefore decreases the incoming flux, we can define $ f_{\#}$ as the f-number
to measure the relative aperture of the camera, where $ f_{\#} = \frac{D_A}{F}$.

We can see that the irradiance is only a measure of this relative aperture accross all devices.

\begin{center}
  \includegraphics[scale=.60]{sensor_aperure_comparison.jpeg}
\end{center}
\captionof{figure}{From, Image Acquisition: Handbook of machine vision engineering:, Volume 1 by M.W. Burke,
Effect of aperture in differing systems.}


In image (28.A) we see that sensors with the same focal length, and thus the same magnification.
The sensor with  $f_2$ relative aperture will have 16 times the incoming light flux and therefore,
brightness compared to $f_8$.

In (28.B) we have sensors with different focal lengths and therefore different magnifications.
Due to the larger focal length, the image must be placed further away from the 8" focal length sensor.
Due to the change in distance, the incident light flux and brightness is again only due to
the relative aperture. In this case, the sensor with $ f_2$ will have 4 times the incoming light flux
and brightnesss compared to $ f_4$

\section{Image Sensor}


A key factor in consistent image analysis with images from different devices
and therefore different image sensors is the metrics of the sensor themselves.
Image sensors are generally built upon silicon diodes. The energy bandgap
of silicon 1.1 eV is very well suited for the capturing visible light spectrum (45o nm -650nm), which
have the photon energies of 2.75eV to 1.9eV. The current line up of smartphones almost universally imply
Back-Side Illuminated (BSI)
CMOS
 (complementary metal oxide semiconductors) architecture.


\subsection{Integration Time or Shutter Speed}

The measure of light hitting the sensor, we must measure the current generated
by the photon hitting the sensor. The output voltage measured by the photodiode,
is given by,

\begin{equation}
  V_{\text{out}} = V_{\text{photo}} - \frac{i_{\text{photo}}t_{\text{int}}}{C_{D}}[1]
\end{equation}
$V_{\text{out}}$ is the output voltage and $V_{\text{photo}} $ is a known
potential to aid in voltage detection.
$ i_{\text{photo}}$ is the photocurrent, $t_{\text{int}}$ is the integration time,
and $C_D $ is the capacitance of the photodiode.

The capacitance and current are usually in the same order (femto), so the integration
time $ t_{\text{int}}$ is linearly proportional to the voltage measured. The integration
time is related to the shutter speed and at faster speeds we have the linear relationship
with intensity. To normalize intensity captured across devices, we can define a model
integration time, $ t_0$ and say,

\begin{equation}
  I \propto \frac{T}{T_0}
\end{equation}
where, T is the integration time or shutter speed used for the image.

\subsection{Pixel Metrics}
Image sensors essentially convert photons to charges. The variance in the
sensors ability to convert photons would cause a variance in the image measurement.

The current generation from photons is not a fully efficient mechanism.
The \textbf{Quantum Efficiency} measures the efficiency of photon conversion
for particular wavelengths. In our particular case, the efficiency at wavelengths
of high absorbance for chlorophyll and CDOM can be significantly different across
devices. The amount of photons being being detected by the sensor would be essentially a
linear function of the integration time or shutter speed and the efficiency at some
wavelength. We can define a constant the efficiency, $ n_0$ for any wavelength between
$ 0 \leq n_0 \leq 100$. So we can say that

\begin{equation}
  I \propto \frac{n_{\lambda}}{n_{0\lambda}}
\end{equation}
Where, $ n_{0\lambda}$ is our model efficiency and $n_{\lambda}$
is the efficiency for a sensor at wavelength, $ \lambda$.



\subsection{Pixel Size}

The pixel size of a sensor referes to the physical single single absorbing photons. The
larger the sensor size, the larger the pixel size and more light the pixel can absorb. The
intensity of light absorbed is therefore proportional to the surface area covered by the pixel
or the square of the pixel size..
\section{Smartphone comparisons}
The overall light gathering capabilities of the sensor is defined by the ratio of the focal length
to the effective aperture of the system. We define that as the f number of the system.
We also have a variance in the sensor size and pixel size on various smartphones. The
pixel size has a direct squared relationship with irradiance.

We choose a baseline model smartphone with an Irradiance Factor (IF) of 1,
The irradiance factor of a sensor is computed by the ratio of the Pixel size (PS) squared over
the Aperutre/ F number squared. The focal length (FL) as discussed above, is included in the f-number.

\begin{table}[h!]
\centering
\begin{tabular}{||c c c c c c||}
 \hline
 Model & IS (MP) & Aperture & PS($\mu$m) & FL(mm) & IF\\ [0.5ex]
 \hline\hline
 Baseline                  & 12 &1.0& 1.00 & 35 &1 \\
 Galaxy S8+                & 12 &1.7& 1.40 & 26 &0.68\\
 iPhone 6S                 & 12 &2.2& 1.22 & 29 &0.31\\
 iPhone 6S Plus            & 12 &2.2& 1.22 & 29 &0.31\\
 iPhone 7                  & 12 &1.8& 1.22 & 28 &0.46\\
 iPhone 7 Plus (WA)        & 12 &1.8& 1.22 & 28 &0.46\\
 iPhone 7 Plus (Telephoto) & 12 &2.8& 1.00 & 56 &0.13\\
 iphone 8                  & 12 &1.8& 1.22 & 28 &0.46\\[1ex]
 \hline
\end{tabular}
\caption{Specifications and Irradiance factor for widely used smartphones}
\label{table:1}
\end{table}

Given the IF, we can see the intensity of light collected by each sensor given the same
environment and shutter speed.

\begin{center}
  \includegraphics[scale=.45]{normalized_irradiance.png}
\end{center}

We can thus transform the pixel data using the IF to obtain consisten data while using
different cameras.
\section{Further Notes}

Up until this point, we have been considering resulting images that were exactly the same. That means
we have been changing sensor positions from the image plane (distance from image), according to the
sensor focal length to obtain identical images. Images taken from varying distances must be normalized
by their relative magnification and therefore, we must establish a baseline magnificaiton value.

-Will add more smartphones for comparison.

-Research ISO ranges and shutter speed ranges for various Smartphones

-Most of the data is from third party so reliability is an issue.

\section{Image Analysis}

The raw data obtained from the camera provides us with the three values in the
R,G,B channels per pixel. Given a specific wavelength, $ \lambda$, we want to obtain the
\underline{intensity} at that wavelenth.

\textbf{This section depends highly on the extraction of pixels from images and further analysis of the pixels. It is quite vague
and will require further work}

\subsection{Spectral Sensitivity Function}

Given constant ISO and aperture, at each wavelenth, $ \lambda$, the spectral sensitivity
in RGB channels is calculated by
$$ c(\lambda) = \frac{d(\lambda)}{r(\lambda) t(\lambda)} $$
where, $ r(\lambda)$ is the radiance. $t(\lambda) $ is the exposure time.
$d(\lambda)$ is the data.

Therefore rdiance at a wavelength would be,

$$ r(\lambda) = \frac{d(\lambda)}{c(\lambda)t'(\lambda, f)} $$

with $t'(\lambda) $ is a function of the wavelength and aperture as well.

\subsection{RGB Image Analysis}

For a Lambartian diffuser, the most general theoretical digital value for each pixel, $ g_i$,

\begin{align}
  g_i = \alpha_i
  \left[\int_{\lambda}I_{\lambda}O_{\lambda} L_{r}F_{i\lambda}C_{\lambda} d\lambda\right]^{\gamma_{i}} + \beta_{i}
\end{align}
where i = {Red, Green, Blue}. $O_{\lambda}$ is the spectral reflectance of the object (in our case the secchi disk),
$C_{\lambda}$ is the spectral sensitivity of the image sensor at the particular wavelength.
$F_{\lambda}$ is the spectral transmittance of the color filters. $ \gamma_i$ is the gamma
correction done by each channel and $ \beta_{i}$ is the error factor due to dark current

If no processing is done by the camera electronics during the photo we can set
$ \gamma_i = \alpha_i = F_{i\lambda} =1$ we get the equation of the form,

\begin{align}
  g_i = \int_{\lambda} I_{\lambda}O_{\lambda} L_{r}C_{\lambda} d\lambda +O(\text{error})
\end{align}

The incident light on the seonsor for any wevalength is given by, $I_{\lambda} O_{\lambda} L$.

Without post-processig the pixel value of each color channel after submerging the disk is given by
\begin{align}
    g_i = \int_{\lambda} I_{\lambda}O_{\lambda} L^{`}_{r}C_{\lambda} d\lambda +O(\text{error})
\end{align}

We notice that the only value that changes is the attentuation function relating Incident intenisty
and light absorption.

The spectral response of the seons $ C_\lambda$ is non-linear not readily available.
If we consider corresponding picels of images of unsubmerged and submerged Secchi
diskds, $ g$ and $g^`$, we can obtain the relative inensities.

Given all other spectral corrections (absorption in other wavelenths especially) are taken into account,
we can see that as an approximation we can say that the relative intensity due attentuation can be
measured by the relative values in each color channel.
\begin{align}
\frac{g_i^`}{g_i} = \frac{L^`_{s}}{L_{s}}
\end{align}

Where, $ L^`_{s}$is the intensity of the light after submersion and $ L_{s}$ is the detected light before submersion.

(Look at Appendix)

\textbf{Notes: }
\begin{itemize}
  \item The integral over the sensor spectral response is essentially a sum over discretized wavelength
  points. So we can do the cancellations for unabsorbed wavelenghts.
  \item The assumption that there is minimal absoprtion in the other spectra is correct only for water without
  any other particulates. A better approximation would require fine tuning in lake water.
\end{itemize}
\section{Related Works}

\subsection{Smartphone Flourescence Spectroscopy}
Yu, Tan, and Cunningham demonstrate a procedure to use smartphones for
spectrophotometry of flourescence based biolological assays. Ther device
involves a collecting lense and diffraction grating to "manually" seperate
the incoming wavelength. Although their setup is designed for mobile lab use, and
would not be applicable for field testing,the image analysis provides and result show
the relative competence of smartphone cameras in detecting visible light for analysis.

Image processing for the flourescence photospectrometer involved only a 850x180 band of pixel
values from a incandescent lamp (150W halogen fiber-optic high intensity illuminator; Cole Parmer, IL, USA).
Interestingly, they obsersved intensity fall offs at two particular wavelengths, 510 nm (cyan) and 580 nm (yellow),
even though the incandescent light should produce continuous intensity. A light
source fed through a tube with a diffraction grating could serve as a quick tool to
measure relative efficiencies of sensors.

The image analysis after obtaining the diffracted images was done after the RGB values
were transformed to Hue-Saturation-Value (HSV) to obtain photon intensities at
particular wavelength bands. This could possibly be a good avenue to find intensities
from RGB in our measurements (After some normalization of course).


\url{http://citeseerx.ist.psu.edu/viewdoc/download?doi=10.1.1.717.9937&rep=rep1&type=pdf}


% \underline{Note:} The intensity of light in the water will lower than
% the intensity of light outside. The refractive be taken into account in the
% rendering equation.
% (The incident light will need to take into account Fresnel Diffraction possibly?)
\section{Appendix}
\subsection{Concentration Function calculation}
The taylor expansion of $ 10^{\frac{-2\epsilon d C}{cos(\theta)}}$ is given by
$$ 10^{\frac{-2\epsilon d C}{ cos(\theta)}} = 10^{-2\epsilon d C} - 10^{-2\epsilon d C}\epsilon d C x^2ln(10)$$
We can use the taylor expantion to solve for rendering equation analytically up to the second order,
\begin{align*}
  L_{i 663} &= \int^{2\pi}_{\phi_i = 0}\int^{\theta_c}_{\theta_i = 0}
  \frac{\rho_D}{\pi}  L_{s 663} 10^\frac{-2\epsilon d C}{cos(\theta)}cos(\theta_i)sin(\theta_i)d\theta_i d\phi_i
      \\    &=  \int^{2\pi}_{\phi_i = 0}\int^{\theta_c}_{\theta_i = 0}
      \frac{\rho_D}{\pi}  L_{s 663}\left(10^{-2\epsilon d C} - 10^{-2\epsilon d C}\epsilon d C \theta_i^2ln(10)\right)
      cos(\theta_i)sin(\theta_i)d\theta_i d\phi_i
      \\    &= \int^{\theta_c}_{\theta_i = 0}
      2\rho_D  L_{s 663}\left(10^{-2\epsilon d C} - 10^{-2\epsilon d C}\epsilon d C \theta_i^2ln(10)\right)
      cos(\theta_i)sin(\theta_i)d\theta_i
      \\ &= 2\rho_D  L_{s 663}10^{-2\epsilon d C} \left(
      \int^{\theta_c}_{\theta_i = 0}cos(\theta_i)sin(\theta_i)d\theta_i-\epsilon d C ln(10)\int^{\theta_c}_{\theta_i = 0} \theta_i^2 cos(\theta_i)sin(\theta_i)d\theta_i
      \right)
      \\&= 2\rho_D  L_{s 663}10^{-2\epsilon d C} \left(\left. \frac{-\cos^2(\theta_i)}{2}\right \rvert^{\theta_c}_{\theta_0} - \left.
      \left(\epsilon d C ln(10)\right)\frac{2\theta_i\sin\left(2\theta_i\right)+\left(1-2\theta_i^2\right)\cos\left(2\theta_i\right)}{8} \right \rvert^{\theta_c}_{\theta_0}
      \right)
\end{align*}
Defining the angular constats as k and q, we can approximate the Concentration function to the second order,
\begin{align*}
L_{i 663} = 2\rho_D  L_{s 663}10^{-2\epsilon d C} (k - Cq)
\end{align*}

\begin{align*}
\frac{L_{i 663}}{2\rho_D  L_{s 663}} &= 10^{2\epsilon d C}(k - Cq)\\
ln \left( \frac{L_{i 663} }{2\rho_D  L_{s 663}} \right ) &= \epsilon d C\cdot ln(10^{-2})+ln(k - Cq) \\
ln \left( \frac{L_{i 663} }{2\rho_D  L_{s 663}} \right ) +k -1 &= C \\
\frac{ln \left( \frac{L_{i 663} 10^{2\epsilon d} }{2\rho_D  L_{s 663}} \right ) +k -1}{q - ln(10)} &= C
\end{align*}

\section{References}
[1] Harold H. Strain, Mary R. Thomas, Joseph J. Katz, Spectral absorption properties of ordinary and fully deuteriated chlorophylls a and b,
In Biochimica et Biophysica Acta, Volume 75, 1963, Pages 306-311,
ISSN 0006-3002, https://doi.org/10.1016/0006-3002(63)90617-6. \newline
% (http://www.sciencedirect.com/science/article/pii/0006300263906176)
[2] https://www.cs.princeton.edu/~smr/cs348c-97/surveypaper.html \newline
[3] https://www2.cs.arizona.edu/classes/cs433/fall08/lectures/BRDFandGlobalIllum.pdf
\newline
[4] Beer-Lambert Law (\url{http://life.nthu.edu.tw/~labcjw/BioPhyChem/Spectroscopy/beerslaw.htm}) \newline
[5] Fundamentals of Image Sensor Performance \url{http://www.cse.wustl.edu/~jain/cse567-11/ftp/imgsens/index.html#sec14}

[6] Image Acquisition: Handbook of machine vision engineering:, Volume 1, M.W. Burke.

[7] Pixel Size \url{https://www.ephotozine.com/article/complete-guide-to-image-sensor-pixel-size-29652}

\subsection{smartphone specs}
[4] \url{https://techcrunch.com/2017/09/22/iphone-8-teardown-reveals-few-surprises-but-more-camera-details/}
[5] \url{https://www.gsmarena.com/}


\end{document}
