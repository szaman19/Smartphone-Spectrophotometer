\documentclass{article}
\usepackage[utf8]{inputenc}
\usepackage{amsmath}
\usepackage{amsfonts}
\usepackage{amssymb}
\usepackage{amsthm}
\usepackage{epsfig}
\usepackage{epstopdf}
\usepackage{titling}
\usepackage{url}
\usepackage{array}
\usepackage{enumerate}
\usepackage{ physics }
\usepackage[hmargin=3.5cm,vmargin=2.5cm]{geometry}
\title{Smartphone Spectrophotometer Notes}
\author{Shehtab Zaman }
\date{December 2017}

\begin{document}
\maketitle
\section{Introduction}

% \section{Background}
\section{Sensor Baseline}

\subsection{Compensating for Exposure}

\subsubsection{Aperture and Intensity}

The aperture of, measured in f-stops, is the opening of the camera sensor. the
f-stop measures the amount of sensors surface open to collect photons. Larger
apertures create a smaller depth of field (should be considered when deciding on the baseline).

The intensity of the incident light is inversely proportional to the square of the
f-number.
$$ I \propto \frac{1}{f^2} $$

\subsubsection{Shutter Speed and Intensity}
We can assume that the shutter speed (how long the sensor collects photons), is inversely proportional
to the measured intensity.

$$ I \propto \frac{1}{T}$$

\subsubsection{Pixel values}

If the pixel value is an estimate of the power per solid angle per area of the sensor, we can combine the
shutter speed and aperture relations as

$$ I \propto \frac{1}{T \cdot f^2}$$    
%\section{Ambient Light Baseline}
\section{Absorbance Model}
\subsection{Lambertian BRDF}
Assuming a diffuse reflection and rotational symmetry of the
secchi disk, we can assume the Lambertian BRDF to be
as constant, with the form,
\[
f(\theta_i, \phi_i; \theta_r , \phi_r) = \frac{\rho_d}{\pi}
\]

The $\rho_d$ is the albedo of the secchi disk.

Inside the water column with the assumption that the light source is normal to the surface,
the surface radiance is then,

\[
L = \frac{\rho_d}{\pi} I_0
\]

Where, $I_0$ is the intensity of the incident light.

\underline{Note:} The intensity of light in the water will lower than
the intensity of light outside.
(The incident light will need to take into account Fresnel Diffraction possibly?)

\section{References}

\begin{thebibliography}{9}
  \bibitem{test}
\end{thebibliography}



\end{document}
