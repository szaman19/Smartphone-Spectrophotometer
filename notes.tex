\documentclass{article}
\usepackage[utf8]{inputenc}
\usepackage{amsmath}
\usepackage{amsfonts}
\usepackage{amssymb}
\usepackage{amsthm}
\usepackage{epsfig}
\usepackage{epstopdf}
\usepackage{titling}
\usepackage{url}
\usepackage{array}
\usepackage{enumerate}
\usepackage{ physics }
\usepackage[hmargin=3.5cm,vmargin=2.5cm]{geometry}
\title{Smartphone Spectrophotometer Notes}
\author{Shehtab Zaman }
\date{December 2017}

\begin{document}
\maketitle
\section{Introduction}

\section{Background}
\section{Sensor Baseline}
\section{Ambient Light Baseline}
\section{Absorbance Model}
\subsection{Lambertian BRDF}
Assuming a diffuse reflection and rotational symmetry of the
secchi disk, we can assume the Lambertian BRDF to be
as constant, with the form,
\[
f(\theta_i, \phi_i; \theta_r , \phi_r) = \frac{\rho_d}{\pi}
\]

The $\rho_d$ is the albedo of the secchi disk.

Inside the water column with the assumption that the light source is normal to the surface,
the surface radiance is then,

\[
L = \frac{\rho_d}{\pi} I_0
\]

Where, $I_0$ is the intensity of the incident light.

\underline{Note:} The intensity of light in the water will lower than
the intensity of light outside.
(The incident light will need to take into account Fresnel Diffraction possibly?)

\section{References}

\begin{thebibliography}{9}
  \bibitem{test}jhjkhjkhjhkj
\end{thebibliography}



\end{document}
