\documentclass{article}
\usepackage[utf8]{inputenc}
\usepackage{amsmath}
\usepackage{amsfonts}
\usepackage{amssymb}
\usepackage{amsthm}
\usepackage{epsfig}
\usepackage{epstopdf}
\usepackage{titling}
\usepackage{url}
\usepackage{array}
\usepackage{enumerate}
\usepackage{ physics }
\usepackage[hmargin=3.5cm,vmargin=2.5cm]{geometry}
\title{Rendering Equation in Water Column}
\author{Shehtab Zaman }
\date{December 2017}

\begin{document}
\section{Figures}
\section{Model}

From the general radiance equation,the reflected radiance is given by

\begin{equation}
  L_r = \int^{2\pi}_{\phi_i = 0}\int^{\frac{\pi}{2}}_{\theta_i = 0}
  f_r L_s T(d)cos(\theta_i)sin(\theta_i)d\theta_i d\phi_i
\end{equation}

Where, the $ f_r$ is Lambertian BRDF.

The transmission function $T(d)$ can be related to the depth of the relfecting
surface.

Given an incident light of radiance, $L_s$,
we have the decomposed radiance as
$$L_{sa} + L_{sb} = L$$
where, $ L_{sa}$ is the absorbed light radiation.



According to the Beer-Lambert Law we have the relation,

\begin{equation}
  Log\left(\frac{L}{L_sb}\right) = T(d)
\end{equation}
and,
\begin{equation}
  T(d) = \frac{\epsilon d C}{cos(\theta)}
\end{equation}
where, $\epsilon$ is the molar absorptivity, $ d$ is the depth, $ C$ is the concentration.

Since we want to specifically take into account the
absorbance due to chlorophyll a, we limit the consideration
to considering the radiance of light at $\lambda = 662nm$

\end{document}
