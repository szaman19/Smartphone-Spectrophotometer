\documentclass{article}
\usepackage[utf8]{inputenc}
\usepackage{amsmath}
\usepackage{amsfonts}
\usepackage{amssymb}
\usepackage{amsthm}
\usepackage{epsfig}
\usepackage{epstopdf}
\usepackage{titling}
\usepackage{url}
\usepackage{array}
\usepackage{enumerate}
\usepackage{ physics }
\usepackage{graphicx}
\usepackage{marginnote}

\graphicspath{ {/} }
\usepackage[hmargin=3.5cm,vmargin=2.5cm]{geometry}
\title{Rendering Equation in Water Column}
\author{Shehtab Zaman }
\date{December 2017}

\begin{document}
% \section{Figures}

\maketitle
\section{Rendering Equation}

The interaction of incident, reflected, and emmited light at any given point is
described by the general Rendering Equation,

\begin{equation}
  L_o = L_e + L_r
\end{equation}
Where, $L_o$ is the light leaving the particular point, $ L_e$ is the light emitted
from the point (if it is a source), and $L_r $ is the light reflected from that point.

Since we will be working with non-light generating disks, we set,
$ L_e = 0$.
The light reflected off of the surface of a disk, can be calculated using the
Bidirectional Reflectance Distribution Function (BRDF)[3].
For a given point on a surface, the general BRDF is,

\begin{equation}
\rho_d(\phi_0, \theta_0,\phi_f, \theta_f)=\frac{L_r{\phi_f, \theta_f}}{L_I{\phi_0, \theta_0}}
\end{equation}
Where $\phi_0, \theta_0$ are the incident azimuthal and polar angle of the light source,
and $\phi_f, \theta_0$ are the azimuthal and polar angle with respect to the viewer and
and relfected point.

From the general rendering equation, the reflected radiance is given by

\begin{equation}
  L_r = \int^{2\pi}_{\phi_i = 0}\int^{\frac{\pi}{2}}_{\theta_i = 0}
  f_r L_s T(d)cos(\theta_i)sin(\theta_i)d\theta_i d\phi_i
\end{equation}

Where, the $ f_r$ is Lambertian BRDF.

The transmission function $T(d)$ can be related to the depth of the reflecting
surface.

\subsection{Beer-Lambert Law}

The Beer-Lambert provides us with a linear relationship betweem incident light, absorbed light,
and concentration of absorbed light.

The wavelenght dependent relation is given as,

\begin{equation}
  \epsilon d c = \Log_10{\frac{I_0}{I_f}}
\end{equation}
where, $\epsilon$ is the molar absorptivity in $ L \cdot cm^{-1} \cdot mol^{-1}$,
$ d$ is the depth in $cm$, $ C$ is the concentration in $mol \cdot L^{-1} $. $ I_0$
is the initial light intensity, and $ I_f$ is the final intensity after
absorption.

Given some incident light of radiance, $L_0$, 
we have the decomposed radiance as
$$L_{sa} + L_{s} = L_0$$
where, $ L_{sa}$ is the absorbed light radiation.

According to the Beer-Lambert Law we have the relation,

\begin{equation}
  Log_{10}\left(\frac{L_0}{L_{s}}\right) = \frac{-\epsilon d C}{cos(\theta)}
\end{equation}
and,
\begin{equation}
  T(d) = 10^\frac{-\epsilon d C}{cos(\theta)}
\end{equation}


Since we want to specifically take into account the
absorbance due to chlorophyll a, we limit the consideration
to considering the radiance of light at $\lambda = 662nm$

Even though the incident light is poloychromatic, but since we are considering the specific wavelength
and the incident light is so spread out, we can reasonably treat it as monochromatic. (?)

The reflectance in the particular frequency then becomes,

\begin{equation}
  L_{r \lambda} = \int^{2\pi}_{\phi_i = 0}\int^{\theta_c}_{\theta_i = 0}
  f_r L_{s\lambda} 10^\frac{-\epsilon d C}{cos(\theta)}cos(\theta_i)sin(\theta_i)d\theta_i d\phi_i
\end{equation}
\marginnote{The cutoff angle should ideally vary with the azimuthal position of the sun, time of day and the lattitude of position.}
Here we sum the incident angle upto some cutoff angle $\theta_c$, as the higher incident lights
have negligible effect on the reflected light.


We can further make an assumption that the secchi disk is perfectly matte with rotationally
invariant reflectance. The Lambertian BRDF is thus,

\begin{equation}
  f_r = \frac{\rho_D}{\pi}
\end{equation}
Where, $ \rho_D$ is the albedo of the secchi disk with $ 0 \leq \rho_D \leq 1$
Combining (4) and (5) we get the reflectance off of the surface of secchi disk as,

\begin{equation}
  L_{r \lambda} = \int^{2\pi}_{\phi_i = 0}\int^{\theta_c}_{\theta_i = 0}
  \frac{\rho_D}{\pi} L_{s\lambda} 10^\frac{-\epsilon d C}{cos(\theta)}cos(\theta_i)sin(\theta_i)d\theta_i d\phi_i
\end{equation}

Chlorophyll a has an extinction coefficient of $ \epsilon_{663} = 0.088 \frac{cm^{-1}}{g}$ at 663nm.
Since the light collected by the camera will have had to travel up the water column we must add
another absorbance factor determined by the equation 2.

The light incident on the camera is therefore,
\begin{align}
  L_{i 663} = \int^{2\pi}_{\phi_i = 0}\int^{\theta_c}_{\theta_i = 0}
  \frac{\rho_D}{\pi}  L_{s 663} 10^\frac{-2\epsilon d C}{cos(\theta)}cos(\theta_i)sin(\theta_i)d\theta_i d\phi_i
\end{align}
\section{Analytic Solution}

\section{Model}

Carrying out a simplistic modelling of the rendering equation with the secchi disk albedo, $ \rho_d = 1$,
we can carry out the model the intensity percent ratio of reflected light absorbed by the camera and incident light on
the submerged secchi disk, $ \frac{L_{r663}}{L_{i 663}}$ with,
\begin{center}
    \includegraphics[scale=.70]{ContourPlot}
\end{center}

The concentration ranges from $ 1 \cdot 10^{-7} mol\cdot L^{-1}$ to $5 \cdot 10^{-7} mol\cdot L^{-1}$. The depth being modeled
range from $10cm$ to $ 50cm$

Holding the concentration consant at $ 1 \cdot 10^{-7} mol\cdot L^{-1}$, we can model the
relationship between $ \frac{L_{r663}}{L_{i663}}$ and depth.
\begin{center}
    \includegraphics[scale=.5]{2dPlot}
\end{center}

Following the same procedure as before, keeping the depth constant at $ 20 cm$, we have the relationship between the
concentration and intensity ratio ($ \frac{L_{r663}}{L_{i663}}$).
\begin{center}
  \includegraphics[scale=.5]{2dPlotConcentration}
\end{center}

The simulations were carried out with numerical approximations in mathemtica and the cutoff angle angle used was $ \frac{\pi}{2.1}$.
($\frac{\pi}{2}$ creates a discontinuity in the integrand)

\section{Notes/Definitions}
\textbf{Radiance:} Radiant power per unit foreshortened area per unit solid angle. "throuput for light given point in a given direction" [3]
\newline
\textbf{Irradiance:} Incident power per unit area. (Light incident on surface or point)
\newline
\textbf{Bidirectinal Reflectance Distribution Function (BRDF):} The ratio of irradiance and radiance on a surface.
\newline
\section{References}

[1]Harold H. Strain, Mary R. Thomas, Joseph J. Katz, Spectral absorption properties of ordinary and fully deuteriated chlorophylls a and b, In Biochimica et Biophysica Acta, Volume 75, 1963, Pages 306-311, ISSN 0006-3002, https://doi.org/10.1016/0006-3002(63)90617-6.
% (http://www.sciencedirect.com/science/article/pii/0006300263906176)

[2]https://www.cs.princeton.edu/~smr/cs348c-97/surveypaper.html \newline
[3] https://www2.cs.arizona.edu/classes/cs433/fall08/lectures/BRDFandGlobalIllum.pdf
\newline
[4] Beer-Lambert Law (http://life.nthu.edu.tw/~labcjw/BioPhyChem/Spectroscopy/beerslaw.htm)
\end{document}
