\documentclass{article}
\usepackage[utf8]{inputenc}
\usepackage{amsmath}
\usepackage{amsfonts}
\usepackage{amssymb}
\usepackage{amsthm}
\usepackage{epsfig}
\usepackage{epstopdf}
\usepackage{titling}
\usepackage{url}
\usepackage{array}
\usepackage{enumerate}
\usepackage{ physics }
\usepackage{graphicx}
\graphicspath{ {/} }
\usepackage[hmargin=3.5cm,vmargin=2.5cm]{geometry}
\title{Rendering Equation in Water Column}
\author{Shehtab Zaman }
\date{December 2017}

\begin{document}
% \section{Figures}
\section{Model}

From the general radiance equation,the reflected radiance is given by

\begin{equation}
  L_r = \int^{2\pi}_{\phi_i = 0}\int^{\frac{\pi}{2}}_{\theta_i = 0}
  f_r L_s T(d)cos(\theta_i)sin(\theta_i)d\theta_i d\phi_i
\end{equation}

Where, the $ f_r$ is Lambertian BRDF.

The transmission function $T(d)$ can be related to the depth of the relfecting
surface.

Given an incident light of radiance, $L_0$,
we have the decomposed radiance as
$$L_{sa} + L_{s} = L_0$$
where, $ L_{sa}$ is the absorbed light radiation.



According to the Beer-Lambert Law we have the relation,

\begin{equation}
  Log_{10}\left(\frac{L_0}{L_{s}}\right) = Log_{10}(T(d))
\end{equation}
and,
\begin{equation}
  T(d) = 10^\frac{-\epsilon d C}{cos(\theta)}
\end{equation}
where, $\epsilon$ is the molar absorptivity, $ d$ is the depth, $ C$ is the concentration.

Since we want to specifically take into account the
absorbance due to chlorophyll a, we limit the consideration
to considering the radiance of light at $\lambda = 662nm$

Even though the incident light is poloychromatic, but since we are considering the specific wavelenth
and the incident light is so spread out, we can reasonably treat it as monochromatic. (?)

The reflectance in the particular frequency then becomes,

\begin{equation}
  L_{r \lambda} = \int^{2\pi}_{\phi_i = 0}\int^{\theta_c}_{\theta_i = 0}
  f_r L_{s\lambda} 10^\frac{-\epsilon d C}{cos(\theta)}cos(\theta_i)sin(\theta_i)d\theta_i d\phi_i
\end{equation}
Here we sum the incident angle upto some cutoff angle $\theta_c$, as the higher incident lights
have negligible effect on the
We can further make an assumption that the secchi disk is perfectly matte with rotationally
invariant reflectance. The Lambertian BRDF is thus,

\begin{equation}
  f_r = \frac{\rho_D}{\pi}
\end{equation}
Combining (4) and (5) we get the reflectance off of the surface of secchi disk as,

\begin{equation}
  L_{r \lambda} = \int^{2\pi}_{\phi_i = 0}\int^{\theta_c}_{\theta_i = 0}
  \frac{\rho_D}{\pi} L_{s\lambda} 10^\frac{-\epsilon d C}{cos(\theta)}cos(\theta_i)sin(\theta_i)d\theta_i d\phi_i
\end{equation}

Plotting the function with varying depth and concentration (ranging form $0<C<1$), we get,

\begin{center}
    \includegraphics[scale=.5]{3dPlot}
\end{center}
Restricting the cutoff to $ \theta_c = \frac{\pi}{3}$, holding the concentration consant,
we obtain the relationship between $ \frac{L_{r\lambda}}{L_{s\lambda}}$ and depth (not in cm as we are using
$ \epsilon =1$ for modelling purposes)
\begin{center}
    \includegraphics[scale=.5]{2dPlot}
\end{center}
\newline




\end{document}
