\documentclass{article}
\usepackage[utf8]{inputenc}
\usepackage{amsmath}
\usepackage{amsfonts}
\usepackage{amssymb}
\usepackage{amsthm}
\usepackage{epsfig}
\usepackage{epstopdf}
\usepackage{titling}
\usepackage{url}
\usepackage{array}
\usepackage{enumerate}
\usepackage{ physics }
\usepackage{graphicx}
\usepackage{marginnote}
\usepackage{getfiledate}
\usepackage{wrapfig}
\usepackage{gensymb}
\usepackage{hyperref}

\graphicspath{ {/} }
\usepackage[hmargin=3.5cm,vmargin=2.5cm]{geometry}
\title{Rendering Equation in Water Column}
\author{Shehtab Zaman \\ Binghamton University}
\date{\today}
\begin{document}
% \section{Figures}

\maketitle
\section{Model}
\subsection{Fresnels Reflection and Transmittance}

If we consider the intensity of the light detected by the sensor, $ L_{sensor}$, we can
effectively decompose the sources into two components: the reflection off of the air-water
interface (air-to-air) $L_{reflected}$, and a proportion of the reflectance off of
the disk underwater, $K \cdot L_{transimitted}$. Where
$ K$ is determined by Rendering equation as the concentration of particulates as a parameter.

Given the ambient atmospheric light intenstiy, $L_{atm}$, by energy considerations we can say
$$L_{atm} = L_{reflected}+L_{transmitted}$$

\subsubsection{Reflectance}
Since the incident light is unpolarized, the reflectance is given
by the average of S-polarized and P-polarized light. As an approximation
for a given incident angle $\theta$, we can use Schlick's approximation to obtain
reflectance R.
$$ R(\theta) = \left(\frac{n_1-n_2}{n_1+n_2}\right)^2 +
 \left(1-\left(\frac{n_1-n_2}{n_1+n_2}\right)^2\right)\left(1-cos(\theta)\right)^5 $$
where, $ n_1$ is the index of refraction of the incident media (air in our case)
and $ n_2$ is the index of refracton of the reflected media (water = 1.33).

Since the reflectance is wavelength independent, we have,

$$ L_{reflected} = R \cdot L_{atm}$$

\subsubsection{Transmittance}
\marginnote{In practice considering the reflection may be trivial as at $ \theta = 0$
the maximum reflectance is $\approx 3\%$ and has a significant effect at higher incident
angles only $(>60 \degree)$.}

We can relate the transmittance $ T$ of the unpolarized light with
the reflectance $ R$ as
$$ T = 1 - R$$

We have the transmitted light into the water as,

$$ L_{transmitted} = T \cdot L_{atm} $$

The attenuation $ L_{transmitted}$ due to chlorophyll absoprtion and reflection
on the disk is discussed in the following sections.



\subsection{Rendering Equation}

The interaction of incident, reflected, and emmited light at any given point on a
given surface is
described by the general Rendering Equation,

\begin{equation}
  L_o = L_e + L_r
\end{equation}
Where, $L_o$ is the light leaving the particular point, $ L_e$ is the light emitted
from the point (if it is a source), and $L_r $ is the light reflected from that point.

Since we will be working with non-light generating disks, we set,
$ L_e = 0$.
The light reflected off of the surface of a disk, can be calculated using the
Bidirectional Reflectance Distribution Function (BRDF)[3].
For a given point on a surface, the general BRDF is,

\begin{equation}
\rho_d(\phi_0, \theta_0,\phi_f, \theta_f)=\frac{L_r{\phi_f, \theta_f}}{L_I{\phi_0, \theta_0}}
\end{equation}
Where $\phi_0, \theta_0$ are the incident azimuthal and polar angle of the light source,
and $\phi_f, \theta_0$ are the azimuthal and polar angle with respect to the viewer and
and relfected point.

From the general rendering equation, the reflected radiance is given by

\begin{equation}
  L_r = \int^{2\pi}_{\phi_i = 0}\int^{\frac{\pi}{2}}_{\theta_i = 0}
  f_r L_s T(d)cos(\theta_i)sin(\theta_i)d\theta_i d\phi_i
\end{equation}

Where, the $ f_r$ is Lambertian BRDF.

The transmission function $T(d)$ can be related to the depth of the reflecting
surface.

\subsection{Beer-Lambert Law}

The Beer-Lambert provides us with a linear relationship betweem incident light, absorbed light,
and concentration of absorbed light.

The wavelength dependent relation is given as,

\begin{equation}
  \epsilon d c = Log_{10}{\frac{I_0}{I_f}}
\end{equation}
where, $\epsilon$ is the molar absorptivity in $ L \cdot cm^{-1} \cdot mol^{-1}$,
$ d$ is the depth in $cm$, $ C$ is the concentration in $mol \cdot L^{-1} $. $ I_0$
is the initial light intensity, and $ I_f$ is the final intensity after
absorption.

Given some incident light of radiance, $L_0$, on the water surface,

we have the decomposed radiance as
$$ L_0 = L_{sa} + L_{s}$$
where, $ L_{sa}$ is the absorbed light radiation. $ L_{s}$ is the light
incident on the point under water and being reflected by the disk.

According to the Beer-Lambert Law we have the relation,

\begin{equation}
  Log_{10}\left(\frac{L_0}{L_{s}}\right) = \frac{\epsilon d C}{cos(\theta)}
\end{equation}
and,
\begin{equation}
  T(d) = 10^\frac{-\epsilon d C}{cos(\theta)}
\end{equation}

\subsection{Reflectance}

Since we want to specifically take into account the
absorbance due to chlorophyll a, we limit the consideration
to considering the radiance of light at $\lambda = 662nm$

Even though the incident light is poloychromatic, but since we are considering the specific wavelength
and the incident light is so spread out, we can reasonably treat it as monochromatic. (?)

The reflectance in the particular frequency then becomes,

\begin{equation}
  L_{r \lambda} = \int^{2\pi}_{\phi_i = 0}\int^{\theta_c}_{\theta_i = 0}
  f_r L_{s\lambda} 10^\frac{-\epsilon d C}{cos(\theta)}cos(\theta_i)sin(\theta_i)d\theta_i d\phi_i
\end{equation}
\marginnote{The cutoff angle should ideally vary with the azimuthal position of the sun, time of day and the lattitude of position.}
Here we sum the incident angle upto some cutoff angle $\theta_c$, as the higher incident lights
have negligible effect on the reflected light.


We can further make an assumption that the secchi disk is perfectly matte with rotationally
invariant reflectance. The Lambertian BRDF is thus,

\begin{equation}
  f_r = \frac{\rho_D}{\pi}
\end{equation}
Where, $ \rho_D$ is the albedo of the secchi disk with $ 0 \leq \rho_D \leq 1$
Combining (4) and (5) we get the reflectance off of the surface of secchi disk as,

\begin{equation}
  L_{r \lambda} = \int^{2\pi}_{\phi_i = 0}\int^{\theta_c}_{\theta_i = 0}
  \frac{\rho_D}{\pi} L_{s\lambda} 10^\frac{-\epsilon d C}{cos(\theta)}cos(\theta_i)sin(\theta_i)d\theta_i d\phi_i
\end{equation}

Chlorophyll a has an extinction coefficient of $ \epsilon_{663} = 0.088 \frac{cm^{-1}}{g}$ at 663nm.
Since the light collected by the camera will have had to travel up the water column we must add
another absorbance factor determined by the equation 2.

The light incident on the camera is therefore,
\begin{align}
  L_{i 663} = \int^{2\pi}_{\phi_i = 0}\int^{\theta_c}_{\theta_i = 0}
  \frac{\rho_D}{\pi}  L_{s 663} 10^\frac{-2\epsilon d C}{cos(\theta)}cos(\theta_i)sin(\theta_i)d\theta_i d\phi_i
\end{align}
\subsection{Analytical Solution}
\subsubsection{At High Cutoff Angles}
At high cutoff angles $ \theta_c > 60 \degree$, we can reasonably account for reflectance off
of the water surface to be sufficiently low $ (\leq 3\%)$. The correction to the incident is
sufficiently low and can be discounted.

Equation (10) can be solved analytically to the first order approximation of the taylor expansion of
$10^{\frac{-\epsilon dC}{cos(\theta_i)}}$, for a light source which is at angle $\theta_0$ to $ \theta_{c}$.
The concentration of the chlorophyll, c, can be retrieved by,
\begin{align}
  C &= \frac{ln \left (\frac{L_{i663}}{2\rho_dL_{s663} \left(\left. \frac{\cos^2(\theta_i)}{2} \right \rvert_{\theta_0}^{\theta_c}\right) } \right)}{\epsilon d ln(10^{-2})}
\end{align}

\begin{center}
  \includegraphics[scale=.60]{FirstOrderConcentration.jpg}
\end{center}
Notice that as $d \rightarrow 0 $ the equation reduces to a function of incident angle and properties
of the material and will correspond to the concentration $ C \rightarrow 0 $. The coupling
of $ \epsilon$ and $d$ also shows that in a particulate free medium $(\epsilon = 0)$, the
reflectance is affected by the depth $ d$ and is analogous to the case where $ d = 0$ for
any depth.

In this case, $ \theta_c$ is less than the $ \frac{\pi}{3}$ so we can assume the
difference transmitted intensity is the same as the incident atmospheric intensity. The addition
of higher incident angles would make equation unsolvable analytically
(But I believe we can approximate using perturbation theory).
\subsection{Second Order approximation}
(I'm still grappling with the math on this one)
The Appendix has the math carried out but the last approximation is troubling me.

The orange line is the correction to the first order approximation in the concentraiton
function. The end points behave as expected.
\begin{center}
\includegraphics[scale=.60]{SecondOrderConcentration.jpg}
\end{center}

\subsection{Possible Concerns}

\begin{itemize}
  \item The issues with non-constant air-water interface (waves) is that both the incident
  light intenstiy and reflected light intensity can very dramatically over the the
  surface of the secchi disk and result in patches of high reflectance and distorted disk
  image. Imposing a periodic boundary condition, by defining the water surface as a plane
  wave could be a better approximation. There would be a significant increase in calculation
  complexity in that case.
  \item Another siginificant ommitance is back scattering from the reflected light, being reflected
  back onto the disk. (For an approximation, I believe the correction due to this would be
  marginal, but I have to carry out a sample calculation to have a concrete value).
\end{itemize}

\section{Sensor Baseline}

\subsection{Compensating for Exposure}

\subsubsection{Aperture and Intensity}

The aperture of, measured in f-stops, is the opening of the camera sensor. the
f-stop measures the amount of sensors surface open to collect photons. Larger
apertures create a smaller depth of field (should be considered when deciding on the baseline).

The intensity of the incident light is inversely proportional to the square of the
f-number.
$$ I \propto \frac{1}{f^2} $$

\subsubsection{Shutter Speed and Intensity}
We can assume that the shutter speed (how long the sensor collects photons), is inversely proportional
to the measured intensity.

$$ I \propto \frac{1}{T}$$

\subsubsection{Pixel values}

If the pixel value is an estimate of the power per solid angle per area of the sensor, we can combine the
shutter speed and aperture relations as

$$ I \propto \frac{1}{T \cdot f^2}$$
%\section{Ambient Light Baseline}
\section{Wavelength Intensity}

The raw data obtained from the camera provides us with the three values in the
R,G,B channels per pixel (see mathemtica for example). Given a specific wavelength, $ \lambda$, we want to obtain the
\underline{intensity} at that wavelenth.

\subsection{Spectral Sensitivity Function}

Given constant ISO and aperture, at each wavelenth, $ \lambda$, the spectral sensitivity
in RGB channels is calculated by
$$ c(\lambda) = \frac{d(\lambda)}{r(\lambda) t(\lambda)} $$
where, $ r(\lambda)$ is the radiance. $t(\lambda) $ is the exposure time.
$d(\lambda)$ is the data.

Therefore rdiance at a wavelength would be,

$$ r(\lambda) = \frac{d(\lambda)}{c(\lambda)t'(\lambda, f)} $$

with $t'(\lambda) $ is a function of the wavelength and aperture as well.

\subsection{RGB Image Analysis}
For a Lambartian diffuser, the most general theoretical digital value for each pixel, $ g_i$,

\begin{align}
  g_i = \alpha_i
  \left[\int_{\lambda}I_{\lambda}O_{\lambda} L_{r}F_{i\lambda}C_{\lambda} d\lambda\right]^{\gamma_{i}} + \beta_{i}
\end{align}
where i = {Red, Green, Blue}. $O_{\lambda}$ is the spectral reflectance of the object (in our case the secchi disk),
$C_{\lambda}$ is the spectral sensitivity of the image sensor at the particular wavelength.
$F_{\lambda}$ is the spectral transmittance of the color filters. $ \gamma_i$ is the gamma
correction done by each channel and $ \beta_{i}$ is the error factor due to dark current

If no processing is done by the camera electronics during the photo we can set
$ \gamma_i = \alpha_i = F_{i\lambda} =1$ we get the equation of the form,

\begin{align}
  g_i = \int_{\lambda} I_{\lambda}O_{\lambda} L_{r}C_{\lambda} d\lambda +O(\text{error})
\end{align}

The incident light on the seonsor for any wevalength is given by, $I_{\lambda} O_{\lambda} L$.

Without post-processig the pixel value of each color channel after submerging the disk is given by
\begin{align}
    g_i = \int_{\lambda} I_{\lambda}O_{\lambda} L^{`}_{r}C_{\lambda} d\lambda +O(\text{error})
\end{align}

We notice that the only value that changes is the attentuation function relating Incident intenisty
and light absorption.

The spectral response of the seons $ C_\lambda$ is non-linear not readily available.
If we consider corresponding picels of images of unsubmerged and submerged Secchi
diskds, $ g$ and $g^`$, we can obtain the relative inensities.

Given all other spectral corrections (absorption in other wavelenths especially) are taken into account,
we can see that as an approximation we can say that the relative intensity due attentuation can be
measured by the relative values in each color channel.
\begin{align}
\frac{g_i^`}{g_i} = \frac{L^`_{s}}{L_{s}}
\end{align}

Where, $ L^`_{s}$is the intensity of the light after submersion and $ L_{s}$ is the detected light before submersion.

(Look at Appendix)

\textbf{Notes: }
\begin{itemize}
  \item The integral over the sensor spectral response is essentially a sum over discretized wavelength
  points. So we can do the cancellations for unabsorbed wavelenghts.
  \item The assumption that there is minimal absoprtion in the other spectra is correct only for water without
  any other particulates. A better approximation would require fine tuning in lake water.
\end{itemize}
\section{Experiment}

\subsection{Set Up}
Using a lamp as a point source, we can measure the efficacy of the model. The set up for the experiment is
with a secchi disk positioned submerged at depth $d$, in a solution of water and particulates - possible food coloring
or something similar- with known absorption coefficients $ \epsilon$. We can use a lamp as the point source of light. Placing
the lamp at height $ H$ above the secchi disk, at an angle $\theta_a$.

\section{Numerical Simulations}

Carrying out a simplistic modelling of the rendering equation with the secchi disk albedo, $ \rho_d = 1$,
we can carry out the model the intensity percent ratio of reflected light absorbed by the camera and incident light on
the submerged secchi disk, $ \frac{L_{r663}}{L_{i 663}}$ with,
\begin{center}
    \includegraphics[scale=.40]{ContourPlot}
\end{center}

The concentration ranges from $ 1 \cdot 10^{-7} mol\cdot L^{-1}$ to $5 \cdot 10^{-7} mol\cdot L^{-1}$. The depth being modeled
range from $10cm$ to $ 50cm$

Holding the concentration consant at $ 1 \cdot 10^{-7} mol\cdot L^{-1}$, we can model the
relationship between $ \frac{L_{r663}}{L_{i663}}$ and depth.
\begin{center}
    \includegraphics[scale=.4]{2dPlot}
\end{center}

Following the same procedure as before, keeping the depth constant at $ 20 cm$, we have the relationship between the
concentration and intensity ratio ($ \frac{L_{r663}}{L_{i663}}$).
\begin{center}
  \includegraphics[scale=.4]{2dPlotConcentration}
\end{center}

The simulations were carried out with numerical approximations in mathemtica and the cutoff angle angle used was $ \frac{\pi}{2.1}$.
($\frac{\pi}{2}$ creates a discontinuity in the integrand)
\section{Appendix}
\subsection{Concentration Function calculation}
The taylor expansion of $ 10^{\frac{-2\epsilon d C}{cos(\theta)}}$ is given by
$$ 10^{\frac{-2\epsilon d C}{ cos(\theta)}} = 10^{-2\epsilon d C} - 10^{-2\epsilon d C}\epsilon d C x^2ln(10)$$
We can use the taylor expantion to solve for rendering equation analytically up to the second order,
\begin{align*}
  L_{i 663} &= \int^{2\pi}_{\phi_i = 0}\int^{\theta_c}_{\theta_i = 0}
  \frac{\rho_D}{\pi}  L_{s 663} 10^\frac{-2\epsilon d C}{cos(\theta)}cos(\theta_i)sin(\theta_i)d\theta_i d\phi_i
      \\    &=  \int^{2\pi}_{\phi_i = 0}\int^{\theta_c}_{\theta_i = 0}
      \frac{\rho_D}{\pi}  L_{s 663}\left(10^{-2\epsilon d C} - 10^{-2\epsilon d C}\epsilon d C \theta_i^2ln(10)\right)
      cos(\theta_i)sin(\theta_i)d\theta_i d\phi_i
      \\    &= \int^{\theta_c}_{\theta_i = 0}
      2\rho_D  L_{s 663}\left(10^{-2\epsilon d C} - 10^{-2\epsilon d C}\epsilon d C \theta_i^2ln(10)\right)
      cos(\theta_i)sin(\theta_i)d\theta_i
      \\ &= 2\rho_D  L_{s 663}10^{-2\epsilon d C} \left(
      \int^{\theta_c}_{\theta_i = 0}cos(\theta_i)sin(\theta_i)d\theta_i-\epsilon d C ln(10)\int^{\theta_c}_{\theta_i = 0} \theta_i^2 cos(\theta_i)sin(\theta_i)d\theta_i
      \right)
      \\&= 2\rho_D  L_{s 663}10^{-2\epsilon d C} \left(\left. \frac{-\cos^2(\theta_i)}{2}\right \rvert^{\theta_c}_{\theta_0} - \left.
      \left(\epsilon d C ln(10)\right)\frac{2\theta_i\sin\left(2\theta_i\right)+\left(1-2\theta_i^2\right)\cos\left(2\theta_i\right)}{8} \right \rvert^{\theta_c}_{\theta_0}
      \right)
\end{align*}
Defining the angular constats as k and q, we can approximate the Concentration function to the second order,
\begin{align*}
L_{i 663} = 2\rho_D  L_{s 663}10^{-2\epsilon d C} (k - Cq)
\end{align*}

\begin{align*}
\frac{L_{i 663}}{2\rho_D  L_{s 663}} &= 10^{2\epsilon d C}(k - Cq)\\
ln \left( \frac{L_{i 663} }{2\rho_D  L_{s 663}} \right ) &= \epsilon d C\cdot ln(10^{-2})+ln(k - Cq) \\
ln \left( \frac{L_{i 663} }{2\rho_D  L_{s 663}} \right ) +k -1 &= C \\
\frac{ln \left( \frac{L_{i 663} 10^{2\epsilon d} }{2\rho_D  L_{s 663}} \right ) +k -1}{q - ln(10)} &= C
\end{align*}

\subsection{Color Channel Value with Attenuation}



\section{Notes/Definitions}

\textbf{Radiance:} Radiant power per unit foreshortened area per unit solid angle. "throuput for light given point in a given direction" [3]
\newline
\textbf{Irradiance:} Incident power per unit area. (Light incident on surface or point)
\newline
\textbf{Bidirectional Reflectance Distribution Function (BRDF):} The ratio of irradiance and radiance on a surface.
\textbf{Dark Current:} Residual current on camera sensor without any incident light. (related to degradation of sensor possibly?)



\section{References}

[1] Harold H. Strain, Mary R. Thomas, Joseph J. Katz, Spectral absorption properties of ordinary and fully deuteriated chlorophylls a and b,
In Biochimica et Biophysica Acta, Volume 75, 1963, Pages 306-311,
ISSN 0006-3002, https://doi.org/10.1016/0006-3002(63)90617-6. \newline
% (http://www.sciencedirect.com/science/article/pii/0006300263906176)
[2] https://www.cs.princeton.edu/~smr/cs348c-97/surveypaper.html \newline
[3] https://www2.cs.arizona.edu/classes/cs433/fall08/lectures/BRDFandGlobalIllum.pdf
\newline
[4] Beer-Lambert Law (\url{http://life.nthu.edu.tw/~labcjw/BioPhyChem/Spectroscopy/beerslaw.htm}) \newline
[5] Incomplete Gamma Function (\url{http://people.math.sfu.ca/~cbm/aands/page_260.htm})
\end{document}
